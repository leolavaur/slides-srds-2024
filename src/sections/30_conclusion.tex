\section*{Conclusion}

\begin{frame}
  \sectionpage
\end{frame}

\begin{frame}{Fighting Byzantine Attacks in Federated Learning}
  \textbf{The proposed framework can:}
  \begin{itemize}
    \item Leverage cross evaluation, clustering and reputation to address heterogeneity and poisoning.  
    \item Adjust rapidly to changes in behavior.  
    \item Mitigate most tested scenarios. Only one, limit case on colluding targeted attackers poisoning more than 80\% of their data. 
  \end{itemize}


  \pause
  \textbf{How generic?}
  \begin{itemize}
    \item Only few conditions: parametric models, locally owned evaluation set, a \alert<3>{small-scale use case}, and a \alert<3>{trusted central server}.
  \end{itemize}
\end{frame}


\begin{frame}{Future Work}
  \textbf{Future works:}
    \begin{itemize}
      \item Remove the central server dependency.
      \item Reduce the cross-evaluation overhead to extend applicability to cross device settings.
      \item Test the approach in more realistic heterogeneous settings.
    \end{itemize}
\end{frame}


\begin{frame}
  \centering\scshape\large Thank you for your attention!

  \vfill
  
  \normalshape\normalsize

  % \textbf{RADAR: Model Quality Assessment for Reputation-aware\\ Collaborative Federated Learning}
  % \medskip
  
  \raggedright
  \begin{itemize}
    \item A novel architecture for cross-silo FL, combining reputation and clustering to ensure model quality.
    % \item An evaluation showing great coverage in the mitigated scenarios.
    \item An evaluation highlighting \texttt{RADAR} resilience against a great range of failure and attack scenarios.
    \item Promising research directions towards trustable, decentralized, privacy-preserving collaborative learning.
  \end{itemize}

  \vfill

  \centering\small
  leo.lavaur@imt-atlantique.fr\\
  pierre-marie.lechevalier@imt-atlantique.fr

\end{frame}



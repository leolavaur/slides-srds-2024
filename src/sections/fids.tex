\section{FIDS Definition and Limitations}

\begin{frame}
  \sectionpage

  \fcitefootnote{lavaur_tnsm_2022}
  \fcitefootnote{lavaur_cesar_2022}
  \fcitefootnote{lavaur_icdcs_demo_2024}
\end{frame}


\begin{frame}{State of the Art}

  \begin{columns}
    \begin{column}{0.5\textwidth}

      \textbf{Systematic Literature Review (SLR)}~\autocite{lavaur_tnsm_2022}
      \begin{itemize}
        \item Booming topic over the last 4--5 years.
        \item<2-> Very heterogeneous venues and communities.
        \item<3-> Often considered not as a topic, but\dots
        \begin{itemize}
          \item \dots{}as an application domain for FL;.
          \item \dots{}as a tool for collaborative IDSs (CIDSs) in different use cases (e.g., IT networks, IoT, AVs).
        \end{itemize}
      \end{itemize}
      
    \end{column}

    \begin{column}{0.5\textwidth}
      \begin{figure}
        \centering
        \only<1>{%
          \resizebox{.9\linewidth}{!}{\input{figures/fids/year_histogram.pgf}}
          % Word broken by hand because LaTeX didn't for some reason
          \caption{Evolution of the number of publications on FIDSs (until 2024-04).}    
        }
        \only<2->{%
          \resizebox{\linewidth}{!}{\input{figures/fids/domain_pie.pgf}}
          \caption{Distribution of application domains.}
        }
      \end{figure}
    \end{column}
      
  \end{columns}

  \fcitefootnote{lavaur_tnsm_2022}

\end{frame}

\begin{frame}{Federated Intrusion Detection Systems (FIDSs)}

  \begin{block}{Definition}
    \textit{Distributed IDSs with privacy-preserving federated knowledge.}
    FIDSs leverage FL or similar distributed learning techniques to share and aggregate the models trained locally with other members of the federation.
  \end{block}

  \begin{itemize}
    \item<2-> Taxonomy along four axes: \textbf{Data}, \textbf{Local Operation}, \textbf{Federation}, and \textbf{Aggregation}~\autocite{lavaur_tnsm_2022}.
    \item<2-> Reference architecture for FIDSs.
    
    % \item<3-> Training can happen offline on labelled data, online, or with a combination of both.

    % \item<4-> Federations can be closed (i.e., all participants are identified and trusted) or open (i.e., participants can join and leave the federation at any time).
    
  \end{itemize}

  \only<2->{\fcitefootnote{lavaur_tnsm_2022}}

\end{frame}

\begin{frame}{Limitations of FIDSs}
  
  \textbf{Challenges from the SLR}~\cite{lavaur_tnsm_2022}
  \begin{itemize}
    \item Functionality: \emph{performance}, \emph<1>{\alert<2>{heterogeneity}}, \emph<1>{\alert<2>{transferability}}, \emph{self-defense} and \emph{self-healing}.
    \item Deployment: \emph{adaptability} and \emph{scalability}.
    \item Security and reliability: \emph{security}, \emph{privacy}, \emph<1>{\alert<2>{trust}} and \emph<1>{\alert<2>{reputation}}.
    \item Experimentation: \emph<1>{\alert<2>{evaluation}}.
  \end{itemize}

\end{frame}

% \begin{frame}{Heterogeneity Headaches \normalshape{}\small(bis)}
    
%   \textbf{Case Study}: Collaborative IDSs between different organizations (cross-silo FL).
%   \only<2->{%
%   \begin{itemize}
%     \item Four datasets of the litterature: CIC-CSE-IDS2018, UNSW-NB15, Bot-IoT and ToN\_IoT.
%     \item Different ways of distributing data with various degrees of heterogeneity.
%     \begin{itemize}
%       \item 1 dataset IID (CIC-CSE-IDS2018).
%       \item 1 dataset NIID along labels (CIC-CSE-IDS2018).
%       \item 1 dataset per client.
%     \end{itemize}
%   \end{itemize}
%   }

%   \only<3->{%
%     \textbf{Conclusions}~\cite{lavaur_icdcs_demo_2024}
%   }
%     \begin{itemize}
%       \item<3-> \alert{Yes}, heterogeneity between participants increases performance if the datasets are complementary.
%       \item<4-> \alert{Yes}, FL can propagate knowledge across datasets with different attack classes.
%       \item<5-> \alert{No}, independently generated datasets cannot be aggregated \emph{as-is}.
%     \end{itemize}

%   \only<3->{\fcitefootnote{lavaur_icdcs_demo_2024}}
% \end{frame}

